\documentclass[conference]{IEEEtran}
\IEEEoverridecommandlockouts
% The preceding line is only needed to identify funding in the first footnote. If that is unneeded, please comment it out.
\usepackage{cite}
\usepackage{amsmath,amssymb,amsfonts}
\usepackage{algorithmic}
\usepackage{graphicx}
\usepackage{textcomp}
\usepackage{xcolor}
\def\BibTeX{{\rm B\kern-.05em{\sc i\kern-.025em b}\kern-.08em
    T\kern-.1667em\lower.7ex\hbox{E}\kern-.125emX}}
\begin{document}

\title{Electrogardiograma ECG}

\author{
\IEEEauthorblockN{1\textsuperscript{st} Sergio Carracedo Rodríguez}
\IEEEauthorblockA{\textit{Instrumentación Biomédica} \\
\textit{Universitat Politècnica de València}\\
Valencia, España \\
scarrod@teleco.upv.es}
\and
\IEEEauthorblockN{2\textsuperscript{rd} Jorge Huertas Pastor}
\IEEEauthorblockA{\textit{Instrumentación Biomédica} \\
\textit{Universitat Politècnica de València}\\
Valencia, España \\
jorhuepa@teleco.upv.es}
}

\maketitle

\begin{abstract}
En esta práctica se ha trabajado alrededor del ECG. Caracterizando un dispositivo que realiza esta función y diseñando un programa en LabVIEW capaz de mostrar por pantalla tanto la señal del ECG realizado como las pulsaciones por minuto (ppm) que tiene la persona en cuestión.
\end{abstract}

\section{Introduction}
Esta práctica se ha dividido en dos partes, una de ellas consiste en caracterizar un utilizado para la realización del ECG y posteriormente utilizado para la capatura de la señal mediante la placa NI myRIO y su muestra en el PC.

\section{Amplificador de ECG}
\subsection{Respuesta en frecuencia sin filtro notch}

\subsection{Respuesta en frecuencia con filtro notch}

\subsection{Funcionamiento}

\section{Derivaciones electrocardiográficas estándar}
\subsection{Derivación estándar I sin tercer electrodo}

\subsection{Derivación estándar I con tercer electrodo}

\section{Sistema de adquisición de señales electrocardiográficas}
\subsection{Adquisición de derivaciones electrocardiográficas estándar}

\subsection{Exportación de señales adquiridas}

\subsection{Detector de QRS digital}

\subsection{Pulsómetro}


\end{document}
