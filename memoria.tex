\documentclass[conference]{IEEEtran}
\IEEEoverridecommandlockouts
% The preceding line is only needed to identify funding in the first footnote. If that is unneeded, please comment it out.
\usepackage{cite}
\usepackage{amsmath,amssymb,amsfonts}
\usepackage{algorithmic}
\usepackage{graphicx}
\usepackage{textcomp}
\usepackage{xcolor}
\usepackage{pgfplots}
\usepackage{csvsimple}
\usepackage{tikz}
\usepackage{filecontents}

\def\BibTeX{{\rm B\kern-.05em{\sc i\kern-.025em b}\kern-.08em
    T\kern-.1667em\lower.7ex\hbox{E}\kern-.125emX}}
\begin{document}

\title{Electrogardiograma ECG}

\author{
\IEEEauthorblockN{1\textsuperscript{st} Sergio Carracedo Rodríguez}
\IEEEauthorblockA{\textit{Instrumentación Biomédica} \\
\textit{Universitat Politècnica de València}\\
Valencia, España \\
scarrod@teleco.upv.es}
\and
\IEEEauthorblockN{2\textsuperscript{nd} Jorge Huertas Pastor}
\IEEEauthorblockA{\textit{Instrumentación Biomédica} \\
\textit{Universitat Politècnica de València}\\
Valencia, España \\
jorhuepa@teleco.upv.es}
}

\maketitle

\begin{abstract}
En esta práctica se ha trabajado alrededor del ECG. Caracterizando un dispositivo que realiza esta función y diseñando un programa en LabVIEW capaz de mostrar por pantalla tanto la señal del ECG realizado como las pulsaciones por minuto (ppm) que tiene la persona en cuestión.
\end{abstract}

\section{Introducción}
Esta práctica se ha dividido en dos partes, una de ellas consiste en caracterizar un utilizado para la realización del ECG y posteriormente utilizado para la capatura de la señal mediante la placa NI myRIO y su muestra en el PC.

\section{Amplificador de ECG}
\subsection{Respuesta en frecuencia sin filtro notch}
En la gráfica siguiente se muestra la tensión pico-pico $V_{pp}$ de la señal conforme va variando la frecuencia. En este caso el filtro noctch está \textit{desactivado}. 
\newline

\begin{figure}[htbp]
    \begin{tikzpicture}
        \begin{axis}[xmode=log, 
            grid=both,
            xlabel={$f$ (Hz)},
            ylabel={G (dB)}]
        \addplot table [x=f, y=g, col sep=comma] {g_sinNotch.csv};
        \end{axis}
        \end{tikzpicture}
    \caption{Ganancia sin filtro notch}
    \label{g_sinNotch}
    \end{figure}

En esta gráfica se puede observar la respuesta plana en la banda que estamos observando (desde los 0.5 Hz a los 100 Hz).

\subsection{Respuesta en frecuencia con filtro notch}
En la gráfica siguiente se muestra la tensión pico-pico $V_{pp}$ de la señal conforme va variando la frecuencia. En este caso el filtro noctch está \textit{activado}.
\newline
 
\begin{figure}[htbp]
    \begin{tikzpicture}
        \begin{axis}[xmode=log, 
            grid=both,
            xlabel={$f$ (Hz)},
            ylabel={G (dB)}]
        \addplot table [x=f, y=g, col sep=comma] {g_conNotch.csv};
        \end{axis}
        \end{tikzpicture}
    \caption{Ganancia con filtro notch}
    \label{g_conNotch}
    \end{figure}

Se observa que la tensión cae conforme la frecuencia toma valores cercanos a los 50 Hz, que es la componente que queremos eliminar y que nos introduce la red eléctrica.

Podemos decir también que, observando la respuesta en frecuencia que tendría el integrado AD8232, se asemeja bastante.

\subsection{Funcionamiento}
Para mostrar su funcionamiento se han hecho unas capturas de pantalla al osciloscopio en el que se muestran la señal sin filtro notch y la señal con él, respectivamente.

\begin{figure}[htbp]
    \centerline{\includegraphics[width=0.5\textwidth]{./figures/señal_sinNotch.jpeg}}
    \caption{Señal sin el filtro notch funcionando}
    \label{s_sinNotch}
    \end{figure}

\begin{figure}[htbp]
    \centerline{\includegraphics[width=0.5\textwidth]{./figures/señal_conNotch.jpeg}}
    \caption{Señal con el filtro notch funcionando}
    \label{s_conNotch}
    \end{figure}

Se obvserva que la tensión pico-pico varía dependiendo de la señal que observemos. Sin el filtro notch vemos que ese valor es mayor debido al ruido.

\section{Derivaciones electrocardiográficas estándar}
\subsection{Derivación estándar I sin tercer electrodo}
 
\begin{figure}[htbp]
    \centerline{\includegraphics[width=0.5\textwidth]{./figures/señal_conNotch.jpeg}}
    \caption{D.E. I sin filtro notch}
    \label{s_conNotch}
    \end{figure}
 
\begin{figure}[htbp]
    \centerline{\includegraphics[width=0.5\textwidth]{./figures/señal_conNotch.jpeg}}
    \caption{D.E. I sin filtro notch}
    \label{s_conNotch}
    \end{figure}

\subsection{Derivación estándar I con tercer electrodo}
\begin{figure}[htbp]
    \centerline{\includegraphics[width=0.5\textwidth]{./figures/señal_conNotch.jpeg}}
    \caption{Señal con el filtro notch funcionando}
    \label{s_conNotch}
    \end{figure}

\begin{figure}[htbp]
    \centerline{\includegraphics[width=0.5\textwidth]{./figures/señal_conNotch.jpeg}}
    \caption{Señal con el filtro notch funcionando}
    \label{s_conNotch}
    \end{figure}
    
\section{Sistema de adquisición de señales electrocardiográficas}
\subsection{Adquisición de derivaciones electrocardiográficas estándar}

\subsection{Exportación de señales adquiridas}

\subsection{Detector de QRS digital}

\subsection{Pulsómetro}


\end{document}
